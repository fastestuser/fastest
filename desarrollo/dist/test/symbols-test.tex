\documentclass{article}
\usepackage{czt}   % oz or z-eves or fuzz styles
\newenvironment{machine}[1]{
    \begin{tabular}{@{\qquad}l}\textbf{\kern-1em machine}\ #1\\ }{
    \\ \textbf{\kern-1em end} \end{tabular} }
\newcommand{\machineInit}{\\ \textbf{\kern-1em init} \\}
\newcommand{\machineOps}{\\ \textbf{\kern-1em ops} \\}

\begin{document}
% Para usar con pruebas.refExprPrinter con la intencion de obtener los codigos ASCII de los simbolos Z

\begin{schema}{Schema1}
a: \num \\
b: \nat \\
c: \arithmos \\
d: \nat_1 \\
dd: \power \num \\
e: \power_1 \num \\
f: \finset_1 \num \\
R: A \rel B \\
F1: A \fun B \\
A : \power \num \\
B: \power \num \\
C: \finset \num \\
F2: A \pfun B \\
F3: A \pinj B \\
F4: A \inj B \\
F5: A \psurj B \\
F6: A \surj B \\
F7: A \bij B \\
F8: A \ffun B \\
F9: A \finj B \\
D: 1 .. 3 \\
\where
A \neq B \\
a \notin A \\
A = \emptyset \\
A \subset B \\
A \subseteq B \\
A \cup B = \{\} \\
A \cap B = \{\} \\
A \setminus B = \{\} \\
A \symdiff B = \{\} \\
A = \bigcup \{\} \\
A = \bigcap \{\} \\
\{a \mapsto b\} = A \\
\dom A = \{\} \\
\ran A = \{\} \\
\id A = \{\} \\
A \comp B = \{\}  \\
A \circ B = \{\}  \\
A \dres B = \{\} \\
A \rres B = \{\} \\
A \ndres B = \{\} \\
A \nrres B = \{\} \\
A \limg B \rimg = \{\} \\
A \oplus B = \{\} \\
\plus R = \{\} \\
\star R = \{\} \\
\disjoint A\\
A \partition B \\
a = \negate 1 \\
a \leq b \\
a \geq b \\
a \div b = 1 \\
a \mod b = 1 \\
\# A = 2 \\

\end{schema}

\begin{schema}{Schema2}
 S1: \seq \num \\
S2: \iseq \num \\
S3: \seq_1 \num \\
\where
\langle 1,2,3 \rangle = S1 \\
S1 \cat S2 = S1 \\
S1 \extract 1 = S1 \\ 
S1 \filter 2 = S1 \\
S1 \prefix S2 \\
S1 \suffix S2 \\
S1 \infix S2 \\
S1 \dcat S2 = S1 \\
\rev S1 = S1 \\
\head S1 = 2 \\
\tail S1 = S2 \\
\last S1 = 4 \\
\front S1 = S2 \\
\squash S1 = S2
\end{schema}

\begin{schema}{Schema3}
 \Delta Schema1 \\
 \Xi Schema2 \\
 \Theta Schema1 \\
a: \num \cross \num \\
A: \power \num
\where
\first a = 1 \land \second a=2 \land \succ 1 = 2 \\
\min A = 2 \\
\max A = 2 \\

\end{schema}








\end{document}

