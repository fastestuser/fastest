\documentclass{article}
\usepackage{z-eves}   % oz or z-eves or fuzz styles
\newenvironment{machine}[1]{
    \begin{tabular}{@{\qquad}l}\textbf{\kern-1em machine}\ #1\\ }{
    \\ \textbf{\kern-1em end} \end{tabular} }


\begin{document}


\begin{schema}{Prueba}
 a: \num \\
 b: \num \\
 b': \num \\
\where
((a=1 \iff a=2) \lor a=3) \implies a=4
\end{schema}


El predicado del esquema anterior, en DNF, es:
\begin{center}
 $ a=1 \land \lnot a=2 \land \lnot a=3 $\\
 $ \lor a=2 \land \lnot a=1 \land \lnot a=3  $\\
 $ \lor a=4$
\end{center}


El predicado del esquema anterior, en CNF, es:
\begin{center}
 $ a=1 \lor a=2 \lor a=4 $\\
 $ a=1 \lor \lnot a=1 \lor a=4 $ \\
 $ \lnot a=2 \lor a=2 \lor a=4 $ \\
 $ \lnot a=2 \lor \lnot a=1 \lor a=4 $\\
 $ \lnot a=3 \lor a=4 $
\end{center}


\end{document}
