\documentclass[a4paper, 12pt]{article}
\usepackage{multicol}
\usepackage[latin1]{inputenc}
\usepackage[dvips]{geometry}
\usepackage{czt}
\usepackage{dirtree}
\usepackage[pdftex,colorlinks,linkcolor=blue]{hyperref}
\usepackage[pdftex]{graphicx}
\usepackage[cm]{fullpage}

\newcommand*{\ra}{\rightarrow}
%\newenvironment{machine}[1]{
%\begin{tabular}{@{\qquad}l}\textbf{\kern-1em machine}\ #1\\ }{
%\\ \textbf{\kern-1em end} \end{tabular} }

\newenvironment{desig}{\begin{list}{}{\setlength{\labelsep}{0cm}\setlength{\labelwidth}{0cm}\setlength{\listparindent}{0cm}\setlength{\rightmargin}{\leftmargin}}}{\end{list}}

\title{Formal Specification and Results of Applying Model-Based Testing to a Process Scheduler}

\author{Maximiliano Cristi�\\ {\tt cristia@cifasis-conicet.gov.ar} \\CIFASIS \\ Universidad Nacional de Rosario \\ Flowgate Security Consulting \\ Argentina}

\date{April, 2009}


\begin{document}

\maketitle

\begin{abstract}
This document describes a Z model of the process scheduler described in \cite{Utting00}. Also, this report documents the results of applying Model-Based Testing (MBT) from the Z model, but only up to the step where abstract test cases are generated (including pruning the test tree). MBT was applied using Fastest, a tool developed co-jointly between CIFASIS\footnote{Centro Internacional Franco-Argentino de Ciencias de la Informaci�n y Sitemas -- Rosario -- Argentina.} and Flowgate Security Consulting\footnote{{\tt http://www.flowgate.net}}.
\end{abstract}

\tableofcontents

\section{The Z Model}

We consider the requirements introduced in \cite{Utting00} in section 6.2 and the B model developed in the same book in section 6.2.2. The following model is more a translation from the B model than a formalization of the requirements set there. We will be rather moderate in our comments, introducing them just to point out the differences of our model with respect to the B model, since the reader could find them in Utting and Legeard's book.

The set of process ID's is introduced as a basic type rather than as an extension as done in \cite{Utting00}. Latter we will define a bound that represents the finiteness of the scheduler's process table.

\begin{zed}
[PID]
\end{zed}

We change the type name $STATE$ by $\mathsf{PSTATUS}$ and in our model it has only two elements.

\begin{zed}
PSTATUS ::= waiting | ready
\end{zed}

\begin{zed}
EXCEPTION ::= \\
  \t1 ok | memoryFull | alreadyExists | busyProcess | unknownProcess | noActiveProcess
\end{zed}

\begin{zed}
BIN ::= yes | no
\end{zed}

Since Z does not have an operator like B's $\mathsf{ANY}$ we introduce the following which will remain underspecified. The idea behind $chose$ is that it picks an arbitrary element of the set it receives; there is no nondeterminism in $chose$'s definition. Besides, when MBT will be performed it will be included as a state variable because Fastest does not support axiomatic definitions, yet.

%\begin{axdef}
%chose: \power PID \fun PID
%\end{axdef}

Since the scheduler's process table has room for six processes we define a bound and set it to six as was done by Utting and Legeard when they define the set $\mathsf{PID}$.

%\begin{axdef}
%maxProcs: \nat
%\where
%maxProcs = 6
%\end{axdef}

Opposed to the B model developed by Utting and Legeard ours has more state variables because we want to make some states more explicit. $procs$ will hold the set of processes in $waiting$ or $ready$ status; $current$ will record the active process and $idle$ will tell whether there is an active process or the computer is idle, regardless of the value of $current$.

\begin{schema}{Scheduler}
procs: PID \pfun PSTATUS \\
current: PID \\
idle: BIN
\end{schema}

%\begin{schema}{SchedulerInv}
%Scheduler
%\where
%idle = no \implies current \notin \dom procs \\
%\# \dom procs \leq maxProcs
%\end{schema}

\begin{schema}{SchedulerInit}
Scheduler
\where
procs = \emptyset \\
idle = yes
\end{schema}

\begin{schema}{NewOk}
\Delta Scheduler \\
p?:PID \\
rep!:EXCEPTION
\where
p? \notin \dom procs \\
procs' = procs \cup \{p? \mapsto waiting\} \\
current' = current \\
idle' = idle \\
rep! = ok
\end{schema}

\begin{schema}{NewE1}
\Xi Scheduler \\
p?:PID \\
maxProcs: \nat \\
rep!:EXCEPTION
\where
p? \in \dom procs \\
\# \dom procs < maxProcs \\
maxProcs = 6 \\
rep! = alreadyExists
\end{schema}

\begin{schema}{NewE2}
\Xi Scheduler \\
maxProcs: \nat \\
rep!:EXCEPTION
\where
\# \dom procs = maxProcs \\
maxProcs = 6 \\
rep! = memoryFull
\end{schema}

\begin{zed}
New == NewOk \lor NewE1 \lor NewE2
\end{zed}

\begin{schema}{DelOk}
\Delta Scheduler \\
p?:PID \\
rep!:EXCEPTION
\where
p? \in \dom (procs \nrres \{waiting\}) \\
procs' = \{p?\} \ndres procs \\
current' = current \\
idle' = idle \\
rep! = ok
\end{schema}

\begin{schema}{DelE1}
\Xi Scheduler \\
p?:PID \\
rep!:EXCEPTION
\where
p? \notin \dom procs \\
rep! = unknownProcess
\end{schema}

\begin{schema}{DelE2}
\Xi Scheduler \\
p?:PID \\
rep!:EXCEPTION
\where
p? \in \dom procs \\
procs~p? = waiting \\
rep! = busyProcess
\end{schema}

\begin{zed}
Del == DelOk \lor DelE1 \lor DelE2
\end{zed}

\begin{schema}{ReadyOk1}
\Delta Scheduler \\
p?:PID \\
rep!:EXCEPTION
\where
idle = no \\
p? \in \dom procs \\
procs~p? = waiting \\
procs' = procs \oplus \{p? \mapsto ready\}\\
current' = current \\
idle' = idle \\
rep! = ok
\end{schema}

\begin{schema}{ReadyOk2}
\Delta Scheduler \\
p?:PID \\
rep!:EXCEPTION
\where
idle = yes \\
p? \in \dom procs \\
procs~p? = waiting \\
procs' = \{p?\} \ndres procs \\
current' = p? \\
idle' = no \\
rep! = ok
\end{schema}

\begin{schema}{ReadyE1}
\Xi Scheduler \\
p?:PID \\
rep!:EXCEPTION
\where
p? \notin \dom procs \\
rep! = unknownProcess
\end{schema}

\begin{schema}{ReadyE2}
\Xi Scheduler \\
p?:PID \\
rep!:EXCEPTION
\where
p? \in \dom procs \\
procs~p? \neq waiting \\
rep! = busyProcess
\end{schema}

\begin{zed}
Ready == ReadyOk1 \lor ReadyOk2 \lor ReadyE1 \lor ReadyE2
\end{zed}

\begin{schema}{SwapOk1}
\Delta Scheduler \\
chose: \power PID \fun PID \\
rep!:EXCEPTION
\where
idle = no \\
ready \in \ran procs \\
procs' = 
  \{chose(\dom(procs \rres \{ready\}))\} \ndres procs 
  \cup \{current \mapsto waiting\}\\
current' = chose(\dom(procs \rres \{ready\})) \\
idle' = idle \\
rep! = ok
\end{schema}

\begin{schema}{SwapOk2}
\Delta Scheduler \\
rep!:EXCEPTION
\where
idle = no \\
ready \notin \ran procs \\
procs' = procs \cup \{current \mapsto waiting\}\\
idle' = yes \\
current' = current \\
rep! = ok
\end{schema}

\begin{schema}{SwapOk3}
\Xi Scheduler \\
rep!:EXCEPTION
\where
idle = yes \\
rep! = ok
\end{schema}

\begin{zed}
Swap == SwapOk1 \lor SwapOk2 \lor SwapOk3
\end{zed}

\begin{schema}{DisplayOk1}
\Xi Scheduler \\
ap!, wp!, rp!: \power PID 
\where
idle = no \\
ap! = \{current\} \\
wp! = \dom (procs \rres \{waiting\}) \\
rp! = \dom (procs \rres \{ready\})
\end{schema}

\begin{schema}{DisplayOk2}
\Xi Scheduler \\
ap!, wp!, rp!: \power PID 
\where
idle = yes \\
ap! = \emptyset \\
wp! = \dom (procs \rres \{waiting\}) \\
rp! = \dom (procs \rres \{ready\})
\end{schema}

\begin{zed}
Display == DisplayOk1 \lor DisplayOk2
\end{zed}


\section{Results of Applying MBT to the Scheduler}

\section{Conclusions and Future Work}


\bibliographystyle{alpha}
\bibliography{/home/mcristia/flowgate/institucional/biblio}


\end{document}
