\documentclass[a4paper,11pt]{article}
\usepackage[utf8]{inputenc}

\usepackage[spanish]{babel}


\newenvironment{scaption}[1]{\caption{{\small #1}}}{}

\newenvironment{desig}{\begin{list}{}{\setlength{\labelsep}{0cm}\setlength{\labelwidth}{0cm}\setlength{\listparindent}{0cm}\setlength{\rightmargin}{\leftmargin}}}{\end{list}}


\author{Pablo Rodriguez Monetti}
\title{Diagrama Canónico del sistema Fastest}
\date{2007}

\begin{document}
 \maketitle
El sistema Fastest fue desarrollado sobre una arquitectura Cliente-Servidor, con la intención de aprovechar de manera eficiente los recursos provistos por una red de computadoras. Por un lado, lo que se persigue es obtener mayor performance ejecutando varias tareas en paralelo. Por otro, poder correr más de un cliente del Fastest en distintas terminales, donde todos ellos consultan a un único repositorio de datos.

\section{Las capas lógicas}
\subsection{Lógica de presentación (LP)}
Esta es la capa que se encarga de interactuar en forma directa con el usuario del sistema. Contiene el código responsable de dar disposisición a los elementos gráficos de la pantalla así como de escribir información en la misma, manejar las ventanas y capturar eventos de los distintos dispositivos (por ejemplo, el mouse y el teclado), etc. Esta porción del sistema se ubica exclusivamente en los clientes.
\subsection{Lógica de Negocio (LN)}
Es la parte del sistema que usa los datos de entrada y realiza tareas de negocio. En el caso de Fastest, la misma está dividida entre los clientes y los servidores de cálculo. En los primeros se encontrarán aquellas tareas relativas a:
\begin{itemize}
\item el chequeo de la validez de los datos de entrada, esto es: especificación y código fuente del programa a testear, listado de operaciones a testear, listado de tácticas a aplicar y, eventualmente, nuevas tácticas, funciones de refinamiento y abstracción y reglas de simplificación.
\item la generación del árbol de pruebas partiendo de la especificación y de las tácticas indicadas
\item la proyección de casos de prueba a nivel de especificación a casos de prueba a nivel de implementación y viceversa
\item la ejecución del programa usando los casos de prueba a nivel de implementación
\item guardar casos de prueba abstractos y concretos, así como registro de los tests fallidos y exitosos
\end{itemize}
Por el lado de los servidores de cálculo se encontrará la parte de la lógica de negocio vinvulada a:
\begin{itemize}
 \item la poda de los árboles de prueba generados en los clientes
 \item la construcción de modelos finitos, para poder representar elementos infinitos
 \item la obtención de un caso de prueba para cada clase de prueba, utilizando los modelos finitos previamente generados
 \item la comprobación de la correspondencia entre los casos de prueba a nivel de especificación y la salida del test a nivel de especificación
\end{itemize}
\subsection{Lógica del Procesamiento de los Datos (LPD)}
En esta capa se oculta la forma en que se consultan o almacenan los datos persistentes. dado que tanto clientes como servidores de cálculo hacen uso de tales datos persistentes, hay partes de la LPD en ambos tipos de aplicaciones.
\subsection{DBMS}
En esta capa se almacenan los datos persistentes correspondientes a:
\begin{itemize}
 \item las definiciones de las tácticas de testing
 \item las definiciones de las funciones de refinamiento y abstracción
 \item las reglas de simplificación 
 \item información sobre los servidores de cálculo del sistema
 \item historial de testings realizados previamente
\end{itemize}


\end{document}