%Generales

\begin{theorem}{ExcludedMiddle}{x, \const y, \const z: X}
x = y \\
x = z
\end{theorem}

\begin{theorem}{Reflexivity}{x, y: X}
x \neq y \\
x = y
\end{theorem}

%Indefiniciones

\begin{theorem}{UndefinitionByEmptiness}{f: X \pfun Y}
f = \{ \} \\
\sw( f~\anything ) \\
\end{theorem}

\begin{theorem}{ExtensionalUndefinition}{f: X \pfun Y; x: X}
x \notin \dom f \\
\sw( f~x ) \\
\end{theorem}

%Naturales y enteros

\begin{theorem}{NatDef}{n: \nat}
\lnot 0 \leq n
\end{theorem}

\begin{theorem}{NatDef2}{n: \nat}
n < 0
\end{theorem}

%Aritmetica

\begin{theorem}{ArithmIneq1}{\const N: \nat; n, m: \num}
n \leq m \\
m < N \\
n = N
\end{theorem}

\begin{theorem}{ArithmIneq2}{\const N: \nat; n, m: \num}
n \leq m \\
m < N \\
n > N
\end{theorem}

\begin{theorem}{ArithmIneq3}{\const N: \nat; n, m: \num}
n \leq m \\
m = N \\
n > N
\end{theorem}

\begin{theorem}{ArithmIneq4}{\const N, \const M: \num; n: \num}
\eval( N \leq M ) \\
n \leq N \\
M < n
\end{theorem}

\begin{theorem}{ArithmIneq5}{\const N, \const M: \num; n: \num}
\eval( N \leq M ) \\
n \leq N \\
n = M
\end{theorem}

\begin{theorem}{ArithmIneq6}{\const N, \const M: \num; n: \num}
\eval( M < N ) \\
n > N \\
n = M
\end{theorem}

\begin{theorem}{ArithmIneq7}{\const N, \const M: \num; n: \num}
\eval( M < N ) \\
n > N \\
n < M
\end{theorem}

\begin{theorem}{BasicOrderingProperty}{n, m: \num}
n \neq m \\
\lnot n < m \\
\lnot n > m
\end{theorem}

%Rangos de enteros

\begin{theorem}{NatRangeNotEmpty}{n, \const N, \const M: \nat}
\eval( N \leq M ) \\
n + N \upto ( n + M ) = \{ \}
\end{theorem}

\begin{theorem}{NatRangeNotEmpty2}{\const N, \const M: \nat}
\eval( N < M ) \\
N \upto M = \{ \}
\end{theorem}

\begin{theorem}{NatRangeNotEmpty3}{n, m, \const N, \const M: \nat}
n + ( m * N ) \upto ( n + ( ( m + M ) * N ) ) = \{ \}
\end{theorem}

\begin{theorem}{NatRangeNotSubset}{n, m, \const N, \const M, \const P, \const Q: \nat}
\eval( M - N > Q * P ) \\
N \upto M \subset n + ( m * P ) \upto ( n + ( ( m + Q ) * P ) )
\end{theorem}

\begin{theorem}{NatRangeNotEq}{n, m, \const N, \const M, \const P, \const Q: \nat}
\eval( M - N > Q * P ) \\
N \upto M = n + ( m * P ) \upto ( n + ( ( m + Q ) * P ) )
\end{theorem}

\begin{theorem}{NatRangeNotEmpty5}{n, m, \const P, \const Q: \nat}
\eval( Q * P > 0 ) \\
n + ( m * P ) \upto ( n + ( ( m + Q ) * P ) ) = \{ \}
\end{theorem}

\begin{theorem}{NatRangeNotEmpty4}{n, m: \nat}
n \upto ( n + m ) = \{ \}
\end{theorem}

\begin{theorem}{NatRangeNotSubset2}{n, \const N, \const M, \const P: \nat}
\eval( M - N > P ) \\
N \upto M \subset ( n \upto ( n + P ) )
\end{theorem}

\begin{theorem}{NatRangeNotEq2}{n, \const N, \const M, \const P: \nat}
\eval( M - N > P ) \\
N \upto M = ( n \upto ( n + P ) )
\end{theorem}

\begin{theorem}{NatRangeNotSubset3}{n, m, p, \const N, \const M, \const P, \const Q: \nat}
\eval( M - N > Q * P ) \\
m \leq ( n + Q ) * P \\
N \upto M \subset p + ( n * P ) \upto ( p + m )
\end{theorem}

\begin{theorem}{NatRangeNotEq3}{n, m, p, \const N, \const M, \const P, \const Q: \nat}
\eval( M - N > Q * P ) \\
m \leq ( n + Q ) * P \\
N \upto M = p + ( n * P ) \upto ( p + m )
\end{theorem}

\begin{theorem}{NatRangeNotSubset4}{n, m, \const N, \const M, \const P: \nat}
\eval( M - N > P ) \\
m \leq P \\
N \upto M \subset ( n \upto ( n + m ) )
\end{theorem}

%Conjuntos

\begin{theorem}{SingletonNotSet}{A:\power X; x:X}
x \notin A \\
\se( x ) = A
\end{theorem}

\begin{theorem}{BasicMembershipContradiction}{A:\power X; x:X}
x \in A \\
x \notin A
\end{theorem}

\begin{theorem}{NotInEmptySet}{A:\power X; x:X}
x \in A \\
A = \{ \}
\end{theorem}

\begin{theorem}{SingletonIsNotEmpty}{x:X}
\se( x ) = \{ \}
\end{theorem}

\begin{theorem}{NotSubsetOfSingleton}{A:\power X; x:X}
A \neq \{ \} \\
A \subset \{ x \} 
\end{theorem}

\begin{theorem}{SingletonNotSubset}{A:\power X; x:X}
x \notin A \\
\se( x ) \subset A
\end{theorem}

\begin{theorem}{BasicSetContradiction}{A: \power X}
A = \{ \} \\
A \neq \{ \}
\end{theorem}

\begin{theorem}{ExcludedMiddleSingleton}{A: \power X; \const x, \const y: X}
\{ x \} = A \\
\{ y \} = A
\end{theorem}

\begin{theorem}{ExcludedMiddleSingletonSub1}{A: \power X; \const x, \const y: X}
\{ x \} = A \\
\se( y ) \subset A
\end{theorem}

\begin{theorem}{ExcludedMiddleSingletonSub2}{A: \power X; \const x, \const y: X}
\{ x \} = A \\
A  \subset \{ y \}
\end{theorem}

\begin{theorem}{CapSubsetEmpty}{A, B: \power X}
A \neq \{ \} \\
A \cap B = \{ \} \\
A \subset B
\end{theorem}

\begin{theorem}{CapEqEmpty}{A, B: \power X}
A \neq \{ \} \\
A \cap B = \{ \} \\
A = B
\end{theorem}

\begin{theorem}{CapEmpty}{A, B: \power X}
A = \{ \} \\
A \cap B \neq \{ \}
\end{theorem}

\begin{theorem}{SubseteqSubset}{A, B: \power X}
A \subseteq B \\
B \subset A
\end{theorem}

\begin{theorem}{CapSubseteqEmpty}{A, B: \power X}
A \neq \{ \} \\
A \cap B = \{ \} \\
A \subseteq B
\end{theorem}

\begin{theorem}{NotSubseteqButEqual}{A, B: \power X}
A = B \\
\lnot A \subseteq B
\end{theorem}

\begin{theorem}{SubseteqSubset2}{A, B: \power X}
\lnot A \subseteq B \\
A \subset B
\end{theorem}

\begin{theorem}{EmptySetIsSubseteq}{A, B: \power X}
\lnot A \subseteq B \\
A = \{ \}
\end{theorem}

\begin{theorem}{EmptySetNotSuperset}{A, B: \power X}
A \subseteq B \\
A \neq \{ \} \\
B = \{ \}
\end{theorem}

%Conjuntos por extension

\begin{theorem}{NotinSetExtension}{x, \const y: X}
x \notin \se( y ) \\
x = y
\end{theorem}

%Conjuntos por comprension

\begin{theorem}{SetComprNotEmpty1}{\const N: \num; A: \power X}
\eval( N < 2 ) \\
A \neq \{ \} \\
\{ \anything : N \upto \# A @ \anything \} = \{ \}
\end{theorem}

\begin{theorem}{SetComprNotEmpty2}{A: \power X}
A \neq \{ \} \\
\{ \anything : A @ \anything \} = \{ \}
\end{theorem}

\begin{theorem}{SetComprNotASeq4}{s: \seq X; n: \nat}
n = 0 \\
s \neq \{ \} \\
\dom s = \dom \{ i : 1 \upto \anything @ i + n - 1 \mapsto \anything \}
\end{theorem}

\begin{theorem}{SetComprNotASeq5}{s: \seq X; n: \nat}
n = 0 \\
s \neq \{ \} \\
\dom \{ i : 1 \upto \anything @ i + n - 1 \mapsto \anything \} \subset \dom s
\end{theorem}

\begin{theorem}{SetComprIsEmpty1}{R: X \rel Y}
R = \{ \} \\
\{ \anything : \dom R @ \anything \} \neq \{ \}
\end{theorem}

\begin{theorem}{SetComprIsEmpty2}{A: \power X}
A = \{ \} \\
\{ \anything : A @ \anything \} \neq \{ \}
\end{theorem}

%Relaciones

\begin{theorem}{NotSubsetOfSingletonMapplet}{R: X \rel Y; x:X; y:Y}
R \neq \{ \} \\
\dom R \subset \dom \{ x \mapsto y \} 
\end{theorem}

\begin{theorem}{DomNotSubsetOfSingleton}{R: X \rel Y; x:X}
R \neq \{ \} \\
\dom R \subset \{ x \} 
\end{theorem}

\begin{theorem}{NotInEmptyDom}{R: X \rel Y; x:X}
x \in \dom R \\
R = \{ \}
\end{theorem}

\begin{theorem}{SingletonMappletNotInDom}{R: X \rel Y; x:X; y: Y}
x \notin \dom R \\
\dom \se( x \mapsto y ) = \dom R
\end{theorem}

\begin{theorem}{SingletonNotSubsetDom}{R: X \rel Y; x: X; y: Y}
x \notin \dom R \\
\dom \se( x \mapsto y ) \subset \dom R
\end{theorem}

\begin{theorem}{CardDomEmptyRel}{R: X \rel Y; \const N, n: \nat}
\eval( N > 0 ) \\
R = \{ \} \\
n = N \\
\# \dom R = n
\end{theorem}

\begin{theorem}{CardRelSingleton}{R: X \rel Y; \const N, n: \nat; r: X \cross Y}
\eval( N > 1 ) \\
n = N \\
\# \dom R = n \\
\{ r \} = R
\end{theorem}

\begin{theorem}{SingletonMappletNotEqualRel1}{R: X \rel Y; x: X; y: Y}
x \notin \dom R \\
\se( x \mapsto y ) = R
\end{theorem}

\begin{theorem}{SingletonNotSubsetRel}{R: X \rel Y; x: X; y: Y}
x \notin \dom R \\
\se( x \mapsto y ) \subset R
\end{theorem}

\begin{theorem}{NotInEmptyRan}{R: X \rel Y; y:Y}
y \in \ran R \\
R = \{ \}
\end{theorem}

\begin{theorem}{SingletonMappletNotEqualRel2}{R: X \rel Y; x: X; \const y1, \const y2: Y}
y1 \in \ran R \\
\{ x \mapsto y2 \} = R
\end{theorem}

\begin{theorem}{RanNotSubsetOfSingleton}{R: X \rel Y; y:Y}
R \neq \{ \} \\
\ran R \subset \{ y \} 
\end{theorem}

\begin{theorem}{DomRelCapNotEmpty}{R: X \rel Y; A: \power X}
A \subseteq \dom R \\
\dom R \cap \dom \{ \anything : A @ \anything \} = \{ \}
\end{theorem}

\begin{theorem}{DomRelCapNotSubset}{R: X \rel Y; A: \power X}
A \subseteq \dom R \\
\dom R \subset \dom \{ \anything : A @ \anything \}
\end{theorem}

\begin{theorem}{DomRelCapNotSubset2}{R: X \rel Y; A: \power X}
A \subseteq \dom R \\
\lnot \dom \{ \anything : A @ \anything \} \subseteq \dom R
\end{theorem}

%Restriccion de dominio

\begin{theorem}{DresSubsetEqual}{A:\power X; R: X \rel Y}
R \neq \{ \} \\
A = \dom R \\
R \subset A \dres R 
\end{theorem}

\begin{theorem}{DresSubsetEqual2}{A:\power X; R: X \rel Y}
R \neq \{ \} \\
A = \dom R \\
A \dres R \subset R 
\end{theorem}

\begin{theorem}{DresCapEqual}{A:\power X; R: X \rel Y}
R \neq \{ \} \\
A = \dom R \\
R \cap ( A \dres R ) = \{ \}
\end{theorem}

\begin{theorem}{DresEmpty}{A:\power X; R: X \rel Y}
A = \{ \} \\
A \dres R \neq \{ \}
\end{theorem}

\begin{theorem}{DresEmpty2}{A:\power X; R: X \rel Y}
A = \{ \} \\
\dom ( A \dres R ) \neq \{ \}
\end{theorem}

\begin{theorem}{DresEqual}{A:\power X; R: X \rel Y}
R \neq \{ \} \\
A = \dom R \\
A \dres R = \{ \}
\end{theorem}

%Anti-restriccion de rango

\begin{theorem}{NrresEmptyRel}{R: X \rel Y; A: \power Y; x: X}
x \in \dom (R \nrres A) \\
R = \{ \}
\end{theorem}

\begin{theorem}{NrresCap}{R: X \rel Y; A: \power Y; x: X}
x \in \dom (R \nrres A) \\
\se( x ) \cap \dom R = \{ \}
\end{theorem}

\begin{theorem}{RresEmpty2}{A:\power Y; R: X \rel Y}
A = \{ \} \\
\dom ( R \rres A ) \neq \{ \}
\end{theorem}

\begin{theorem}{RresEmpty3}{A:\power Y; R: X \rel Y}
R = \{ \} \\
\dom ( R \rres A ) \neq \{ \}
\end{theorem}

\begin{theorem}{RresEmpty4}{A:\power Y; R: X \rel Y}
A \cap \ran R = \{ \} \\
\dom ( R \rres A ) \neq \{ \}
\end{theorem}

\begin{theorem}{RresEmpty5}{A:\power Y; R: X \rel Y}
\ran R \subset A \\
\dom ( R \rres A ) \neq \{ \}
\end{theorem}
